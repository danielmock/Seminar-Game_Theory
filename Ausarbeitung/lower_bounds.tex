\section{Lower bounds on the price of malice}

\newthought{How high} can the price of malice be?
In this section we study the havoc a malicious player is able to cause by constructing an example network.
Though a strict upper bound for the price of malice is not yet known.
Which properties of a network may cause a high price of malice?
We take a look at this two main reasons:
\begin{enumerate}
	\item edges that \emph{congest fast} if a small load is added by a malicious player
	\item and \emph{long paths} that a malicious player can choose to increase the load of as many edges as possible.
\end{enumerate}
The first reason is quantifiable by \emph{relative slope}.
Let $\ell \colon [0,1] \to \mathbb R_+$ be a differentiable function.
Its relative slope is defined by \[
rs(\ell) = \max_{x \in [0,1]} \dfrac{\ell'(x)}{\ell(x)}x .\]
Another way to define the relative slope origins from numerical analysis. 
There it is defined by taking the maximum of the relative change in the output $\ell(x)$ compared to the relative change in the input $x$, 
or notate as 
\[ rs(\ell) = \dfrac{\text{relative } \Delta \ell(x)}{\text{relative } \Delta x \phantom{\ell()}} .\]

TODO

\begin{theorem}
	There exists a congestion game with a $d \cdot (m - 1)$ price of malice where the maximal relative slope of the latency functions is $d$ and the network has $\mathcal O(m)$ edges.
\end{theorem}
\begin{proof}
	We construct a congestion game $\mathcal G$ for a given $m$ and a given latency function $\ell$.
	In figure piosfd we see the underlying network for $m=4$.
	The vertex set $E$ consists of the source $s$, the sink $t$ and the vertices $p_1, \dots, p_m$ and $q_1, \dots, q_m$.
	The source has for every $p_i$ an outgoing edge to it and every $p_i$ has an edge to the corresponding $q_i$.
	Every $q_i$ has an edge to the sink and an edge to $p_{i+1}$ (except $q_m$).
	We define the flow value $v$ as $mX$ where $X$ is the point where the relative slope is reached ($\ell'(X)/\ell(X) \cdot X = d$) and the malicious flow value $w$ as $\epsilon m X$ for an $\epsilon$.
	
	Like in examples from previous section, the malicious flow is able to cover every edge with a non-constant latency function with one path, resulting in loading every of those edges as much as possible.
	This \smallcaps{mbr} is independent of the rational flow.
	This and the resulting induced game is pictured in figure TODO.
	Because every direct path $(s, p_i, q_i, t)$ has the latency functions the rational flow of size $(1-\epsilon)mX$ is split up between them. 
	No other paths are chosen for the nash flow because their latency is strictly larger than the latency of the direct ones (like in the previous section).
	Hence, the load on every edge $(p_i, q_i)$ in the middle is $g(e) + f(e)$ which equals $\epsilon m X + (1-\epsilon)X = X+ \epsilon X (m-1)$, thus the latency of every direct path $P$ is $L(P) = \ell(X + \epsilon X (m-1))$ or $L(P)=\ell(X)$ if $\epsilon = 0$ which are the nash delays of the game with ($D(\mathcal G, \epsilon m X)$) and without the malicious player ($D(\mathcal G)$).
	Thereupon we compute the price of malice
	\begin{align*}
		\smallcaps{POM}(\mathcal G) &= \lim_{\epsilon \to 0^+} \dfrac{D(\mathcal G, \epsilon v) - D(\mathcal G)}{\epsilon D(\mathcal G)} \\
		&= \lim_{\epsilon \to 0^+} \dfrac{\ell (X + \epsilon (m-1) X) - \ell(X)}{\epsilon \ell(X)} \\
		&= (m - 1)\dfrac{X \ell'(X)}{\ell(X)} \\
		&= (m - 1) \cdot d
	\end{align*}
	where we use the identity 
	\[ \lim_{\epsilon \to 0^+}\epsilon \to 0 \frac{\ell(x+a\epsilon - \ell(x))}{\epsilon} = a \cdot  \ell'(x) \]
	for $a=X(m-1)$.
\end{proof}

\begin{figure}
	\begin{tikzpicture}
	\SetGraphUnit{2}
	\GraphInit[vstyle=Normal]
	\SetUpEdge[style={->, thick}, color=gray]
	\tikzset{LabelStyle/.style = {fill=none, above}}
	\SetVertexNoLabel
	\Vertex{P1}
	\SO(P1){P2}
	\SO(P2){P3}
	
	\foreach \v in{1,2,3}{\EA[unit=3](P\v){Q\v}}
	
	\SetVertexLabel
	\WE[Lpos=180](P2){s}
	\EA(Q2){t}
	\foreach \v in{1,2,3}{\Edge[label=$\ell(x)$](P\v)(Q\v)}  %  P-Q Kanten mit l(x)
	
	
	\foreach \v in{1,2,3}{\Edge[label=$0$](s)(P\v)}  %  s-P Kanten
	\foreach \v in{1,2,3}{\Edge[label=$0$](Q\v)(t)}  %  Q-t Kanten
	% Rückkanten
	\Edge[label=$0$](Q1)(P2)
	\Edge[label=$0$](Q2)(P3)
	\end{tikzpicture}
	\caption{Congestion game with $m=3$}
	\label{low-base}
\end{figure}
\begin{figure}
	\begin{tikzpicture}
	\SetGraphUnit{2}
	\GraphInit[vstyle=Normal]
	\SetUpEdge[style={->, dashed}, color=gray]
	\tikzset{LabelStyle/.style = {fill=none, above}}
	\SetVertexNoLabel
	\Vertex{P1}
	\SO(P1){P2}
	\SO(P2){P3}
	
	\foreach \v in{1,2,3}{\EA[unit=3](P\v){Q\v}}
	
	\SetVertexLabel
	\WE[Lpos=180](P2){s}
	\EA(Q2){t}
	\foreach \v in{1,2,3}{\Edge(P\v)(Q\v)}  %  P-Q Kanten mit l(x)
	
	
	\foreach \v in{1,2,3}{\Edge(s)(P\v)}  %  s-P Kanten
	\foreach \v in{1,2,3}{\Edge(Q\v)(t)}  %  Q-t Kanten
	% Rückkanten
	\Edge(Q1)(P2)
	\Edge(Q2)(P3)
	
	\Edges[color=red, style={->}](s,P1,Q1,P2,Q2,P3,Q3,t)
	\end{tikzpicture}
	\caption{Malicious best response}
	\label{low-mbr}
\end{figure}
\begin{figure}
	\begin{tikzpicture}
	\SetGraphUnit{2}
	\GraphInit[vstyle=Normal]
	\SetUpEdge[style={->, dashed}, color=gray]
	\tikzset{LabelStyle/.style = {fill=none, above}}
	\SetVertexNoLabel
	\Vertex{P1}
	\SO(P1){P2}
	\SO(P2){P3}
	
	\foreach \v in{1,2,3}{\EA[unit=3](P\v){Q\v}}
	
	\SetVertexLabel
	\WE[Lpos=180](P2){s}
	\EA(Q2){t}
	\foreach \v in{1,2,3}{\Edge[label=$\ell(x+\epsilon m X)$](P\v)(Q\v)}  %  P-Q Kanten mit l(x)
	
	
	\foreach \v in{1,2,3}{\Edge(s)(P\v)}  %  s-P Kanten
	\foreach \v in{1,2,3}{\Edge(Q\v)(t)}  %  Q-t Kanten
	% Rückkanten
	\Edge(Q1)(P2)
	\Edge(Q2)(P3)
	\SetUpEdge[style={->, thick,}, color=blue]
	\foreach \v in{1,2,3}{\Edge(s)(P\v); \Edge(P\v)(Q\v);\Edge(Q\v)(t)}
	\end{tikzpicture}
	\caption{Rational nash flow in the induced game}
	\label{low-rational}
\end{figure}