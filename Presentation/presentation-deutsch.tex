\documentclass{beamer}

\usepackage[euler-digits]{eulervm} % EulerVM math font
\usefonttheme{professionalfonts}
\DeclareMathAlphabet{\mathcal}{OMS}{cmsy}{m}{n} % alte mathcal symbole

\usetheme{metropolis}           % Use metropolis theme

%standard packages
\usepackage[utf8]{inputenc}
\usepackage[ngerman]{babel}
\usepackage{mathtools}
\usepackage{tkz-graph}
\usepackage{booktabs}

\usetikzlibrary{arrows.meta}
%andere packages
% Definitionen
\DeclareMathOperator*{\argmin}{arg\,min}
\DeclareMathOperator*{\argmax}{arg\,max}
%settings
\setsansfont[BoldFont={Fira Sans SemiBold}]{Fira Sans Book} % für shitty Projektoren

% Color Palette by Tol
\definecolor{TolDarkPurple}{HTML}{332288}
\definecolor{TolDarkBlue}{HTML}{6699CC}
\definecolor{myBlue}{HTML}{88CCEE}
\definecolor{myGreen}{HTML}{44AA99}
\definecolor{TolDarkGreen}{HTML}{117733}
\definecolor{TolDarkBrown}{HTML}{999933}
\definecolor{TolLightBrown}{HTML}{DDCC77}
\definecolor{TolDarkRed}{HTML}{661100}
\definecolor{myRed}{HTML}{CC6677}
\definecolor{TolLightPink}{HTML}{AA4466}
\definecolor{TolDarkPink}{HTML}{882255}
\definecolor{TolLightPurple}{HTML}{AA4499}

\definecolor{myBlue}{HTML}{809BC8}
\definecolor{myGreen}{HTML}{64C204}
\definecolor{myRed}{HTML}{FF6666}

% Tikz Set for on-visible
\tikzset{
	invisible/.style={opacity=0},
	visible on/.style={alt={#1{}{invisible}}},
	alt/.code args={<#1>#2#3}{%
		\alt<#1>{\pgfkeysalso{#2}}{\pgfkeysalso{#3}} % \pgfkeysalso doesn't change the path
	},
}

%standard init
\title{Auslastungsspiele mit bösartigen Spielern}
\subtitle{Congestion games with malicious players}
\date{\today}
\author{Daniel A. Mock}
\institute{Lehrstuhl i1 -- RWTH Aachen}

\begin{document}
\begin{frame}{Anmerkungen}
	\begin{itemize}
		\item Wie genau muss ich Auslastungsspiele vorstellen? Wie viel ist bekannt?
		\item Wie ich welche Begriffe übersetze, ist noch offen
		\begin{itemize}
			\item BBR
			\item Delay
			\item malicious = bösartig
			\item Nashfluss
			\item Windfall of Malice
		\end{itemize}
		\item Feintuning bei Farben und Graphenbeschriftungen notwendig
	\end{itemize}
\end{frame}

\maketitle

\setbeamerfont{subsection in toc}{size=\scriptsize}
\setbeamertemplate{section in toc}[sections numbemyRed]

\section{Einführung in Auslastungsspiele}
\begin{frame}{Auslastungsspiel}
	\begin{block}{Szenario}
		\begin{itemize}
			\item $s-t$ Netzwerk $G$ mit Kostenfunktionen für jede Kante
			\item Fluss $f$ mit festen Wert soll von $s$ nach $t$ fließen
			\item Fluss wird von einem Kontinuum von Spielern kontrolliert
			\item Kantenkosten abhängig von der Auslastung dieser Kante
			\item Spieler versuchen Kosten zu optimieren
		\end{itemize}
	\end{block}

	\begin{block}{Später: bösartige Spieler \emph{(malicious players)}}
		\begin{itemize}
			\item ein Teil des Flusses wird von bösartigen Spielern kontrolliert
			\item Ziel: Maximieren der Kosten
		\end{itemize}
	\end{block}
\end{frame}

\begin{frame}{Inhaltsverzeichnis}
	\tableofcontents
\end{frame}

\begin{frame}{Definition Auslastungsspiel}
	Ein symmetrisches, nicht-atomares \alert{Auslastungsspiel} $\mathcal G = (E, \vec{\ell}, \Pi, v)$
	\begin{itemize}
		\item $E$ Kantenmenge
		\item $\vec{\ell}$ Vektor von Latenzfunktionen $\ell_e : \mathbb R_+ \to \mathbb R_+$ für jede Kante $e$
		\item $\Pi$ Menge der $s-t$-Pfade
		\item $v$ erzielter Flusswert
	\end{itemize}
\end{frame}

\begin{frame}{Beispiel}
	\begin{figure}
		\begin{tikzpicture}
		\SetGraphUnit{2}
		\GraphInit[vstyle=Classic]
		\SetUpEdge[style={->, thick}, color=gray]
		\tikzset{LabelStyle/.style = {fill=none, above}}
		\SetVertexNoLabel
		\Vertex{P1}
		\SO(P1){P2}
		
		\foreach \v in{1,2}{\EA[unit=3](P\v){Q\v}}
		
		\SetVertexLabel
		\WE[Lpos=180](P1){s}
		\EA(Q2){t}
		\Edge[label=$x^2$, labelstyle={above}](P1)(Q1)
		\Edge[label=$x$, labelstyle={above}](P2)(Q2)
		
		
		\foreach \v in{1,2}{\Edge[label=$0$](s)(P\v)}  %  s-P Kanten
		\foreach \v in{1,2}{\Edge[label=$0$](Q\v)(t)}  %  Q-t Kanten
		% Rückkanten
		\Edge[label=$0$](Q1)(P2)
		\end{tikzpicture}
	\end{figure}
\end{frame}

\begin{frame}{Fluss}
	Ein \alert{Fluss} ist eine Abbildung $f: \Pi \to \mathbb R^+$. 
	
	Der \alert{Wert} eines Flusses ist \[ |{f}| = \sum_{P \in \Pi} f(P) . \]
	
	Die Menge aller Flüsse mit Wert $v$ im Spiel $\mathcal G$ ist $F(\mathcal G, v)$.
	
	Die Auslastung an einer Kante $e$ ist \[ f(e) = \sum_{\mathclap{P \in \Pi: e \in P}} f(P) . \]
\end{frame}

\begin{frame}{Kosten}
	Der \alert{Delay} auf einem Pfad $P \in \Pi$ ist
	\[ L(P) = \sum_{e \in P} \ell_e\left(f\left(e\right)\right). \]
	
	Die \alert{Kosten} eines Flusses $f$ sind
	\begin{align*}
		C(f) &= \sum_{P \in \Pi} f(P) \cdot L(P) \\
			&= \sum_{e \in E} f(e) \cdot \ell_e\left(f\left(e\right)\right). 
	\end{align*} 
\end{frame}

\begin{frame}{Beispiel}
	Spiel auf dem Graphen mit $v = 1$.
	\begin{figure}
		\begin{tikzpicture}
		\SetGraphUnit{2}
		\GraphInit[vstyle=Classic]
		\SetUpEdge[style={->, thick}, color=gray]
		\tikzset{LabelStyle/.style = {fill=none, above}}
		\SetVertexNoLabel
		\Vertex{P1}
		\SO(P1){P2}
		
		\foreach \v in{1,2}{\EA[unit=3](P\v){Q\v}}
		
		\SetVertexLabel
		\WE[Lpos=180](P1){s}
		\EA(Q2){t}
		\Edge[label=$x^2$, labelstyle={above}](P1)(Q1)
		\Edge[label=$x$, labelstyle={above}](P2)(Q2)
		
		
		\foreach \v in{1,2}{\Edge[label=$0$](s)(P\v)}  %  s-P Kanten
		\foreach \v in{1,2}{\Edge[label=$0$](Q\v)(t)}  %  Q-t Kanten
		% Rückkanten
		\Edge[label=$0$](Q1)(P2)
		
		% Fluss
		\draw [myBlue, ultra thick, ->, visible on=<2->] plot [smooth] coordinates{(s) (P1) (Q1) (P2) (Q2) (t)};
		\draw [myGreen, ultra thick, ->, visible on=<3->, opacity=0.75] plot [smooth] coordinates{(s) (P2) (Q2) (t)};
		\draw [myRed, ultra thick, ->, visible on=<4->, opacity=0.75] plot [smooth] coordinates{(s) (P1) (Q1) (t)};
	\end{tikzpicture}
	\end{figure}
	
	\begin{columns}[T]
		\begin{column}{0.3\textwidth}
			\only<1-5>{
				\textbf{Flusswerte}:
				\begin{itemize}
					\item<2-> $f(\text{\textcolor{myBlue}{blau}}) = 0.1$
					\item<3-> $f(\text{\textcolor{myGreen}{grün}}) = 0.7$
					\item<4-> $f(\text{\textcolor{myRed}{rot}}) = 0.2$
				\end{itemize} }
			
			\only<6->{
				\textbf{Auslastungen}:
				\begin{itemize}
					\item $f(a) = 0.3$
					\item $f(b) = 0.8$
				\end{itemize} }
		\end{column}
		\begin{column}{0.7\textwidth}
			\only<5>{
				\textbf{Auslastungen}:
				\begin{itemize}
					\item $f(a) = 0.1 + 0.2 = 0.3$
					\item $f(b) = 0.7 + 0.1 = 0.8$
			\end{itemize} }
			
			\only<6->{
				\textbf{Latenzen}:
				\begin{itemize}
					\item<6-> $L(\text{\textcolor{myBlue}{blau}}) = f(a)^2 + f(b) = 0.89$
					\item<6-> $L(\text{\textcolor{myGreen}{grün}}) = f(b) = 0.8$
					\item<6-> $L(\text{\textcolor{myRed}{rot}}) = f(a)^2 = 0.09$
				\end{itemize} }
		\end{column}
	\end{columns}

	\only<7->{
		\textbf{Kosten}: \\
		 $C(f)$ = $0.1 \cdot L(\text{blau}) + 0.7 \cdot L(\text{grün}) + 0.2 \cdot L(\text{rot}) = 0.667$ }
\end{frame}

\begin{frame}{Nashfluss}
	Sei $f$ ein gültiger Fluss auf $\mathcal G$. 
	Der Fluss $f$ ist ein \alert{Nash-Fluss}, wenn für alle $P, P' \in \Pi$ und $f(P) > 0$ gilt
	\[
		L(P) \leq L(P').
	\]
	Für einen Nashfluss $f$:
	\begin{itemize}
		\item \emph{Nashkosten} $C(\mathcal G) := C(f)$
		\item \emph{Nashdelay} $D(\mathcal G) := D(f) := C(f)/|f|$
	\end{itemize}
	\pause
	\begin{block}{Beobachtung}
		Jeder Nashfluss hat die gleichen Kosten $C(f)$. \\
		Für alle Pfade $P$ mit $f(P) > 0$ ist $L(P) = D(f)$, wenn $f$ Nashfluss.
	\end{block}
\end{frame}

\begin{frame}{Nashfluss im Beispiel}
	\begin{figure}
		\begin{tikzpicture}
			\SetGraphUnit{2}
			\GraphInit[vstyle=Classic]
			\SetUpEdge[style={->, thick}, color=gray]
			\tikzset{LabelStyle/.style = {fill=none}}
			\SetVertexNoLabel
			\Vertex{P1}
			\SO(P1){P2}
			
			\foreach \v in{1,2}{\EA[unit=3](P\v){Q\v}}
			
			\SetVertexLabel
			\WE[Lpos=180](P1){s}
			\EA(Q2){t}
			\Edge[label=$x^2$, labelstyle={above}](P1)(Q1)
			\Edge[label=$x$, labelstyle={above}](P2)(Q2)
			 %  P-Q Kanten mit l(x)
			
			
			\foreach \v in{1,2}{\Edge(s)(P\v)}  %  s-P Kanten
			\foreach \v in{1,2}{\Edge(Q\v)(t)}  %  Q-t Kanten
			% Rückkanten
			\Edge(Q1)(P2)
			
			% Fluss
			\draw [myRed, ultra thick, ->] plot [smooth] coordinates{(s) (P1) (Q1)  (t)};
			\draw [myGreen, ultra thick, ->] plot [smooth] coordinates{(s) (P2) (Q2) (t)};
		\end{tikzpicture}
	\end{figure}

	\begin{columns}[T]
		\begin{column}{0.5\textwidth}
			Nashfluss $f$:
			\begin{itemize}
				\item $f(\text{blau}) = 0 $
				\item $f(\text{grün}) = \frac{3 - \sqrt{5}}{2} \approx 0.38$
				\item $f(\text{rot}) = \frac{-1 + \sqrt{5}}{2} \approx 0.62 $
			\end{itemize}
		\end{column}
		\begin{column}{0.5\textwidth}
			Delays der Pfade:
			\begin{itemize}
				\item $L(\text{blau}) = 3- \sqrt{5} $
				\item $L(\text{grün}) = \frac{3 - \sqrt{5}}{2}$
				\item $L(\text{rot}) = \frac{3 - \sqrt{5}}{2}$
			\end{itemize}
		\end{column}
	\end{columns}
\end{frame}

\begin{frame}{Bekanntes über Auslastungsspiele}
	\begin{itemize}
		\item sind Potentialspiele
		\item pures Nashequilibrium existiert
	\end{itemize}
\end{frame}


\section{Bösartige Spieler}
\begin{frame}{Bösartige Spieler}
	\begin{itemize}
		\item kontrolliert Fluss $g$ mit Flusswert $w$
		\item maximiert Kosten der rationalen Spieler
		\item möglicherweise randomisiert mit W'keitverteilung $\gamma$
		\item wird als \emph{ein} Spieler modelliert
	\end{itemize}
\end{frame}

\begin{frame}{Beispiel: Bösartiger Fluss}
Spiel mit $v = 1$ und $w = 0.2$ auf diesem Graphen:
\begin{figure}
	\begin{tikzpicture}
	\SetGraphUnit{2}
	\GraphInit[vstyle=Classic]
	\SetUpEdge[style={->, thick}, color=gray]
	\tikzset{LabelStyle/.style = {fill=none, above}}
	\SetVertexNoLabel
	\Vertex{P1}
	\SO(P1){P2}
	
	\foreach \v in{1,2}{\EA[unit=3](P\v){Q\v}}
	
	\SetVertexLabel
	\WE[Lpos=180](P1){s}
	\EA(Q2){t}
	\Edge[label=$x^2$, labelstyle={above}](P1)(Q1)
	\Edge[label=$x$, labelstyle={above}](P2)(Q2)
	
	
	\foreach \v in{1,2}{\Edge[label=$0$](s)(P\v)}  %  s-P Kanten
	\foreach \v in{1,2}{\Edge[label=$0$](Q\v)(t)}  %  Q-t Kanten
	% Rückkanten
	\Edge[label=$0$](Q1)(P2)
	
	% Fluss
	\draw [myRed, ultra thick, ->, visible on=<2->] plot [smooth] coordinates{(s) (P1) (Q1) (t)};
	\end{tikzpicture}
\end{figure}
\end{frame}

\begin{frame}{Beispiel (fortgesetzt)}
	Durch bösartigen Fluss $g$ induziertes Spiel $G^g = (E, \ell^g(x), \Pi, v-w = \textcolor{myRed}{0.8})$:
	\begin{figure}
		\begin{tikzpicture}
		\SetGraphUnit{2}
		\GraphInit[vstyle=Classic]
		\SetUpEdge[style={->, thick}, color=gray]
		\tikzset{LabelStyle/.style = {fill=none, above}}
		\SetVertexNoLabel
		\Vertex{P1}
		\SO(P1){P2}
		
		\foreach \v in{1,2}{\EA[unit=3](P\v){Q\v}}
		
		\SetVertexLabel
		\WE[Lpos=180](P1){s}
		\EA(Q2){t}
		\Edge[label=$(x + 0.2)^2$, labelstyle={myRed}](P1)(Q1) %  P-Q Kanten mit l(x)
		\Edge[label=$x$](P2)(Q2)
		
		\foreach \v in{1,2}{\Edge[label=$0$](s)(P\v)}  %  s-P Kanten
		\foreach \v in{1,2}{\Edge[label=$0$](Q\v)(t)}  %  Q-t Kanten
		% Rückkanten
		\Edge[label=$0$](Q1)(P2)
		\end{tikzpicture}
	\end{figure}
\end{frame}

\begin{frame}{Induziertes Spiel}
	Sei $\mathcal G = (E, \vec{\ell}, \Pi, v)$ und $g$ ein bösartiger Fluss aus $F(\mathcal G,w)$.
	
	Dann ist $\mathcal G^\gamma = (E, \vec{\ell}^\gamma, \Pi, v-w)$ ein \alert{induziertes Spiel} mit 
	\[ \ell_e^\gamma(x) = \ell_e(x+ g(e)) .\]
	
	Auch definierbar für eine W'keitsverteilung $\gamma$ über den Erwartungswert.
\end{frame}


\begin{frame}{Malicious Best Response}
	Für einen Fluss $f$ (der rationalen Spieler) im Spiel $\mathcal G$ ist die \alert{beste bösartige Reaktion} \emph{(malicious best response)} zu $f$ die Menge der Flüsse
	\[ BBR(f) = \argmax_{g \in F(\mathcal G, w)} \, C^g(f) .\]
	
	Eine W'keitsverteilung $\gamma$ auf $F(\mathcal G, w)$ ist ein BBR zu $f$, wenn ein Träger von $\gamma$ in $BBR(f)$ liegt.
\end{frame}

\begin{frame}{Equilibrium}
	Für ein Spiel $\mathcal G$ ist $(f, g)$ ein \alert{pures Equilibrium}, wenn 
	\begin{itemize}
		\item $f$ ist Nashfluss im induziertem Spiel $\mathcal G^g$
		\item $g$ ist BBR zu $f$.
	\end{itemize}
	
	Das Equilibrium ist \alert{semi-pur}, wenn W'keitverteilung $\gamma$ statt Fluss $g$ benutzt wird.
	
	Nashdelay $D(\mathcal G, w)$ ist
	\[
		\sup \left\{\frac{C^g(f)}{|f|} : (f,g) \text{ ist ein Equilibrium} \right\}
	\]
\end{frame}

\begin{frame}{Equilibrium im Beispiel}
	Betrachte Spiel mit $v=1$ und $w=0.2$:
	
	\begin{figure}
		\begin{tikzpicture}
			\SetGraphUnit{2}
			\GraphInit[vstyle=Classic]
			\SetUpEdge[style={->, thick}, color=gray]
			\tikzset{LabelStyle/.style = {fill=none, above}}
			\SetVertexNoLabel
			\Vertex{P1}
			\SO(P1){P2}
			
			\foreach \v in{1,2}{\EA[unit=3](P\v){Q\v}}
			
			\SetVertexLabel
			\WE[Lpos=180](P1){s}
			\EA(Q2){t}
			\Edge[label=$x^2$, labelstyle={above}](P1)(Q1)
			\Edge[label=$x$, labelstyle={above}](P2)(Q2)
			
			
			\foreach \v in{1,2}{\Edge[label=$0$](s)(P\v)}  %  s-P Kanten
			\foreach \v in{1,2}{\Edge[label=$0$](Q\v)(t)}  %  Q-t Kanten
			% Rückkanten
			\Edge[label=$0$](Q1)(P2)
			
			% Fluss
			\draw [myRed, ultra thick, ->, visible on=<1->] plot [smooth] coordinates{(s) (P1) (Q1) (P2) (Q2) (t)};
		\end{tikzpicture}
	\end{figure}
	Für jeden Fluss $f$ ist $g$ eine BBR.
\end{frame}

\begin{frame}{Equilibrium im Beispiel}
	Betrachte induziertes Spiel mit $v-w=0.8$:
	
	\begin{figure}
		\begin{tikzpicture}
		\SetGraphUnit{2}
		\GraphInit[vstyle=Classic]
		\SetUpEdge[style={->, thick}, color=gray]
		\tikzset{LabelStyle/.style = {fill=none, above}}
		\SetVertexNoLabel
		\Vertex{P1}
		\SO(P1){P2}
		
		\foreach \v in{1,2}{\EA[unit=3](P\v){Q\v}}
		
		\SetVertexLabel
		\WE[Lpos=180](P1){s}
		\EA(Q2){t}
		\Edge[label=$(x+0.2)^2$, labelstyle={above}](P1)(Q1)
		\Edge[label=$x+0.2$, labelstyle={above}](P2)(Q2)
		
		
		\foreach \v in{1,2}{\Edge[label=$0$](s)(P\v)}  %  s-P Kanten
		\foreach \v in{1,2}{\Edge[label=$0$](Q\v)(t)}  %  Q-t Kanten
		% Rückkanten
		\Edge[label=$0$](Q1)(P2)
		
		
		% Fluss
		\draw [myGreen, ultra thick, ->] plot [smooth] coordinates{(s) (P1) (Q1)  (t)};
		\draw [myGreen, ultra thick, ->] plot [smooth] coordinates{(s) (P2) (Q2) (t)};
		\end{tikzpicture}
	\end{figure}
	Nashfluss $f$ mit $f(a) \approx 0.5$ und $f(b) \approx 0.3$.
\end{frame}

\begin{frame}{Price of Malice}
	Sei $\mathcal G = (E, \vec{\ell}, \Pi, v)$ ein Spiel mit einem bösartigen Spieler.
	Dann ist der \alert{Price of Malice} definiert durch
	\begin{align*}
		POM(f) &= \lim_{\epsilon \to 0^+} \dfrac{D(\mathcal G, \epsilon v) - D(\mathcal G)}{\epsilon D(\mathcal G)}  \\
		&= \frac{1}{D(\mathcal G)} \cdot \frac{d}{d \epsilon}(D(\mathcal G, v \epsilon))_{(\epsilon = 0)}.
	\end{align*}
\end{frame}

\begin{frame}{Weiteres zu Equilbria}
	Bei Auslastungsspielen mit einem bösartigen Spieler 
	\begin{itemize}
		\item existiert immer ein \emph{semi-pures} Nashgleichgewicht, wobei der bösartige Spieler randomisieren darf
		\item existieren pure Nashgleichgewichte (PNE) \emph{nicht} immer
		\item mit \emph{konkaven} Latenzfunktionen existiert ein PNE
	\end{itemize}
\end{frame}

\section{Untere Schranke für den PoM}

\begin{frame}{Ursachen für einen hohen PoM}
	\begin{itemize}
		\item leicht 'verstopfbare' Kanten: Latenzfunktion steigt sehr schnell
		\item lange Pfade
	\end{itemize}
\end{frame}

\begin{frame}{Relative Slope}
	Sei $\ell : [0,1] \to \mathbb R^+$ differenzierbar.
	Dann ist der \alert{relative Slope} $d$
	\[ d = \sup_{x \in [0,1]} \dfrac{\ell'(x)}{\ell(x)} x .\]
	Dies entspricht dem \emph{relativen Fehler} $\kappa_{rel} = \dfrac{\text{rel. } \Delta y}{\text{rel. } \Delta x}$.
	
	\begin{table}[]
		\begin{tabular}{@{}llll@{}}
			\toprule
			Funktion       & const. & $x^a$ & $e^{ax}$ \\ \midrule
			relative Slope & $0$      & $a$     & $a$        \\ \bottomrule
		\end{tabular}
	\end{table}
	\pause
	\begin{theorem}
		Es gibt ein Spiel $\mathcal G$ mit \alert{PoM $d \cdot (m-1)$} mit $\mathcal O(m)$ Kanten und maximalem relative Slope $d$.
	\end{theorem}
\end{frame}

\begin{frame}{Netzwerk mit hohem PoM}
	\begin{figure}
	\begin{tikzpicture}
		\SetGraphUnit{2}
		\GraphInit[vstyle=Classic]
		\SetUpEdge[style={->, thick}, color=gray]
		\tikzset{LabelStyle/.style = {fill=none}}
		\SetVertexNoLabel
		\Vertex{P1}
		\SO(P1){P2}
		\SO(P2){P3}
		
		\foreach \v in{1,2,3}{\EA[unit=3](P\v){Q\v}}
		
		\SetVertexLabel
		\WE[Lpos=180](P2){s}
		\EA(Q2){t}
		\foreach \v in{1,2,3}{\Edge[label=$\ell(x)$, labelstyle={above}](P\v)(Q\v)}  %  P-Q Kanten mit l(x)
	
		
		\foreach \v in{1,2,3}{\Edge(s)(P\v)}  %  s-P Kanten
		\foreach \v in{1,2,3}{\Edge(Q\v)(t)}  %  Q-t Kanten
		% Rückkanten
		\Edge(Q1)(P2)
		\Edge(Q2)(P3)
	\end{tikzpicture}
	\caption{Beispielnetzwerk mit $m=3$}
	\end{figure}
	
	Flusswert $v(\mathcal G) = mZ$ mit $Z = \argmax \ell'(x)/\ell(x) \cdot x$ \\
	Flusswert vom bösartigen Spieler $w(\mathcal G) = \epsilon v = \epsilon m Z$
\end{frame}

\begin{frame}{Bösartiger Fluss $g$}
	Um die Kosten zu maximieren, nutzt der bösartige Spieler alle Kanten, deren Kostenfunktion nicht konstant ist (Auslastung $\epsilon m Z$):
	\begin{figure}
		\begin{tikzpicture}
		\SetGraphUnit{2}
		\GraphInit[vstyle=Classic]
		\SetUpEdge[style={->, thick}, color=gray]
		\tikzset{LabelStyle/.style = {fill=none}}
		\SetVertexNoLabel
		\Vertex{P1}
		\SO(P1){P2}
		\SO(P2){P3}
		
		\foreach \v in{1,2,3}{\EA[unit=3](P\v){Q\v}}
		
		\SetVertexLabel
		\WE[Lpos=180](P2){s}
		\EA(Q2){t}
		\foreach \v in{1,2,3}{\Edge[labelstyle={above}](P\v)(Q\v)}  %  P-Q Kanten mit l(x)
		
		
		\foreach \v in{1,2,3}{\Edge(s)(P\v)}  %  s-P Kanten
		\foreach \v in{1,2,3}{\Edge(Q\v)(t)}  %  Q-t Kanten
		% Rückkanten
		\Edge(Q1)(P2)
		\Edge(Q2)(P3)
		
		% Bösartiger Fluss
		\draw [myRed, ultra thick, ->] plot [smooth] coordinates{(s) (P1) (Q1) (P2) (Q2) (P3) (Q3) (t)};
		\end{tikzpicture}
	\end{figure}
	Induzierte Latenzfunktion $\ell^g(x) = \ell(x + \epsilon m Z)$
\end{frame}

\begin{frame}{Nashfluss $f$}
	Rationale Spieler kontrollieren $(1- \epsilon)mZ$ Fluss
	\begin{figure}
		\begin{tikzpicture}
		\SetGraphUnit{2}
		\GraphInit[vstyle=Classic]
		\SetUpEdge[style={->, thick}, color=gray]
		\tikzset{LabelStyle/.style = {fill=none, color=gray}}
		\SetVertexNoLabel
		\Vertex{P1}
		\SO(P1){P2}
		\SO(P2){P3}
		
		\foreach \v in{1,2,3}{\EA[unit=3](P\v){Q\v}}
		
		\SetVertexLabel
		\WE[Lpos=180](P2){s}
		\EA(Q2){t}
		\foreach \v in{1,2,3}{\Edge[label=$\ell(x + \epsilon v)$, labelstyle={above}](P\v)(Q\v)}  %  P-Q Kanten mit l(x)
		\foreach \v in{1,2,3}{\Edge(s)(P\v)}  %  s-P Kanten
		\foreach \v in{1,2,3}{\Edge(Q\v)(t)}  %  Q-t Kanten
		% Rückkanten
		\Edge(Q1)(P2)
		\Edge(Q2)(P3)
		
		%Pfad
		\draw [myBlue, ultra thick] plot [smooth] coordinates{(s) (P1) (Q1) (t)};
		\draw [myBlue, ultra thick] plot [smooth] coordinates{(s) (P2) (Q2) (t)};
		\draw [myBlue, ultra thick, ->] plot [smooth] coordinates{(s) (P3) (Q3) (t)};
		\end{tikzpicture}
	\end{figure}
	Auf den blauen Pfaden ist der Fluss $(1- \epsilon)Z$, sonst $0$. Damit ist die Auslastung an den mittleren Kanten: $\textcolor{myBlue}{f(e)} + \textcolor{myRed}{g(e)} = (1 - \epsilon)Z + \epsilon m Z = Z + \epsilon (m-1) Z$
\end{frame}

\begin{frame}{Berechnung PoM}
	Auslastung der mittleren Kanten: $Z + \epsilon Z (m-1)$
	
	Damit ist die Latenz der blauen Pfade $L(P) = \ell (Z + \epsilon (m-1) Z)$.
	Da $f$ der einzige Nashflus ist, ist der Nashdelay $D(\mathcal G, v) = L(P)$.
	Für $\epsilon = 0$ erhalten wir den Nashdelay $D(\mathcal G) = \ell(Z)$.
	\begin{align*}
		POM(\mathcal G) &= \lim_{\epsilon \to 0^+} \dfrac{D(\mathcal G, \epsilon v) - D(\mathcal G)}{\epsilon D(\mathcal G)} \\
			&= \lim_{\epsilon \to 0^+} \dfrac{\ell (Z + \epsilon (m-1) Z) - \ell(Z)}{\epsilon \ell(Z)} \\
			&= \dfrac{(m - 1) Z \ell'(Z)}{\ell(Z)} \\
			&= d \cdot (m - 1)
	\end{align*}
\end{frame}


\section{Paradox der Bösartigkeit}

\begin{frame}{Netzwerk}
	\begin{columns}[T]
		\begin{column}{0.5\textwidth}
			\begin{tikzpicture}[scale=0.7]
				\SetGraphUnit{2}
				\GraphInit[vstyle=Classic]
				\SetUpEdge[style={->, thick}, color=gray]
				\tikzset{LabelStyle/.style = {fill=none, sloped, above}}
				
				% Knoten
				\Vertices[unit=3]{circle}{t,p,s,q}
				
				% Kanten
				\Edge[label=$x/2$](s)(p)
				\Edge[label=$x/2$](q)(t)
				\Edge[label=$1$](s)(q)
				\Edge[label=$1$](p)(t)
			\end{tikzpicture}
			Flusswert $v=1$
		\end{column}
		\pause
		\begin{column}{0.5\textwidth}
			  \begin{tikzpicture}[scale=0.7]
				  \SetGraphUnit{2}
				  \GraphInit[vstyle=Classic]
				  \SetUpEdge[style={->, thick}, color=myGreen]
				  \tikzset{LabelStyle/.style = {fill=none, sloped, above}}
				  
				  % Knoten
				  \Vertices[unit=3]{circle}{t,p,s,q}
				  
				  % Kanten
				  \Edge[label=$x/2$](s)(p)
				  \Edge[label=$x/2$](q)(t)
				  \Edge[label=$1$](s)(q)
				  \Edge[label=$1$](p)(t)
				  
				  % Nashfluss
				  \SetUpEdge[style={->, thick}, color=myGreen]
				  \tikzset{LabelStyle/.style = {below, sloped, fill=none, color=myGreen}}
				  \Edges[label=$1/2$](s,p,t)
				  \Edges[label=$1/2$](s,q,t)
			  \end{tikzpicture}
			  Nashfluss mit $D(f) = 1.25$
		\end{column}
	\end{columns}
\end{frame}

\begin{frame}{Braess' Paradoxon}
	\begin{columns}[T]
		\begin{column}{0.5\textwidth}
			\begin{tikzpicture}[scale=0.7]
				\SetGraphUnit{2}
				\GraphInit[vstyle=Classic]
				\SetUpEdge[style={->, thick}, color=gray]
				\tikzset{LabelStyle/.style = {fill=none, sloped, above, color=gray}}
				
			 Knoten
				\Vertices[unit=3]{circle}{t,p,s,q}
				\EA[Lpos=180](s){m}
				\EA[unit=2](m){n}
				
			 Kanten
				\Edge[label=$x/2$](s)(p)
				\Edge[label=$x/2$](q)(t)
				\Edge[label=$1$](s)(q)
				\Edge[label=$1$](p)(t)
				\Edge[label=$0$](p)(n)
				\Edge[label=$0$](m)(n)
				\Edge[label=$0$](m)(q)
				\Edge[label=$x$](p)(m)
				\Edge[label=$x$](n)(q)
			\end{tikzpicture}
		\end{column}
		\pause
		\begin{column}{0.5\textwidth}
			\begin{tikzpicture}[scale=0.7]
			\SetGraphUnit{2}
			\GraphInit[vstyle=Classic]
			\SetUpEdge[style={->, thick}, color=gray]
			\tikzset{LabelStyle/.style = {fill=none, sloped, above, color=gray}}
			
			% Knoten
			\Vertices[unit=3]{circle}{t,p,s,q}
			\EA[Lpos=180](s){m}
			\EA[unit=2](m){n}
			
			% Kanten
			\Edge(s)(q)
			\Edge(p)(t)
			\Edge(m)(n)
			
			% Nashfluss
			\SetUpEdge[style={->, thick}, color=myGreen]
			\tikzset{LabelStyle/.style = {above, sloped, fill=none, color=myGreen}}
			\Edge[label=$1$](s)(p)
			\Edge[label=$0.5$](p)(m)
			\Edge[label=$0.5$](p)(n)
			\tikzset{LabelStyle/.style = {below, sloped, fill=none, color=myGreen}}
			\Edge[label=$0.5$](m)(q)
			\Edge[label=$0.5$](n)(q)
			\Edge[label=$1$](q)(t)
			\end{tikzpicture}
			Nashfluss mit $D(f) = 1.5$
		\end{column}
	\end{columns}
\end{frame}

\begin{frame}{Umkehrung durch bösartig Spieler}
	\begin{columns}[T]
		\begin{column}{0.5\textwidth}
			\begin{tikzpicture}[scale=0.7]
			\SetGraphUnit{2}
			\GraphInit[vstyle=Classic]
			\SetUpEdge[style={->, thick}, color=gray]
			\tikzset{LabelStyle/.style = {fill=none, sloped, above, color=gray}}
			
			% Knoten
			\Vertices[unit=3]{circle}{t,p,s,q}
			\EA[Lpos=180](s){m}
			\EA[unit=2](m){n}
			
			% Kanten
			\Edge(s)(p)
			\Edge(q)(t)
			\Edge(s)(q)
			\Edge(p)(t)
			\Edge(p)(n)
			\Edge(m)(n)
			\Edge(m)(q)
			\Edge(p)(m)
			\Edge(n)(q)
			
			% Nashfluss
			\SetUpEdge[style={->, thick}, color=myRed]
			\tikzset{LabelStyle/.style = {below, sloped, fill=none, color=myRed}}
			\Edges(s,p,m,n,q,t)
			\end{tikzpicture}
			BBR $g$ mit $w = \epsilon$ \\
			$g$ ''verstopft'' den Pfad des Nashflows
		\end{column}
		\pause
		\begin{column}{0.5\textwidth}
			\begin{tikzpicture}[scale=0.7]
			\SetGraphUnit{2}
			\GraphInit[vstyle=Classic]
			\SetUpEdge[style={->, thick}, color=gray]
			\tikzset{LabelStyle/.style = {fill=none, sloped, above, color=gray}}
			
			% Knoten
			\Vertices[unit=3]{circle}{t,p,s,q}
			\EA[Lpos=180](s){m}
			\EA[unit=2](m){n}
			
			% Kanten
%			\Edge(s)(p)
%			\Edge(q)(t)
%			\Edge(s)(q)
%			\Edge(p)(t)
%			\Edge(p)(n)
			\Edge(m)(n)
%			\Edge(m)(q)
%			\Edge(p)(m)
%			\Edge(n)(q)
			
			% Nashfluss
			\SetUpEdge[style={->, thick}, color=myGreen]
			\tikzset{LabelStyle/.style = {below, sloped, fill=none, color=myRed}}
			\Edges(p,m,q)
			\Edges(p,n,q)
			\SetUpEdge[style={->, ultra thick}, color=myGreen]
			\Edges(s,q,t)
			\Edges(s,p,t)
			\end{tikzpicture}
			Nashflow wird ''nach außen'' gedrängt. \\
			$D(\mathcal G, \epsilon) \to 1.25$
		\end{column}
	\end{columns}
\end{frame}
\end{document}