\documentclass{beamer}

\usepackage[euler-digits]{eulervm} % EulerVM math font
\usefonttheme{professionalfonts}
\DeclareMathAlphabet{\mathcal}{OMS}{cmsy}{m}{n} % alte mathcal symbole

\usetheme{metropolis}           % Use metropolis theme
%standard packages
\usepackage[utf8]{inputenc}
\usepackage[ngerman]{babel}
\usepackage{physics}
\usepackage{mathtools}
\usepackage{tkz-graph}
\usepackage{booktabs}

\usetikzlibrary{arrows.meta}
%andere packages
% Definitionen
\DeclareMathOperator*{\argmin}{arg\,min}
\DeclareMathOperator*{\argmax}{arg\,max}
%settings
\setsansfont[BoldFont={Fira Sans SemiBold}]{Fira Sans Book} % für shitty Projektoren

\tikzset{
	invisible/.style={opacity=0},
	visible on/.style={alt={#1{}{invisible}}},
	alt/.code args={<#1>#2#3}{%
		\alt<#1>{\pgfkeysalso{#2}}{\pgfkeysalso{#3}} % \pgfkeysalso doesn't change the path
	},
}

%standard init
\title{Auslastungsspiele mit bösartigen Spielern}
\subtitle{Congestion games with malicious players}
\date{\today}
\author{Daniel A. Mock}
\institute{Lehrstuhl i1 -- RWTH Aachen}

\begin{document}
\begin{frame}{Anmerkungen}
	\begin{itemize}
		\item Wie genau muss ich Auslastungsspiele vorstellen? Wie viel ist bekannt?
		\item Wie ich welche Begriffe übersetze, ist noch offen
		\begin{itemize}
			\item MBR
			\item Delay
			\item malicious = bösartig
			\item Nashfluss
			\item Windfall of Malice
		\end{itemize}
		\item Overlays fehlen
		\item Beispiele fehlen
	\end{itemize}
\end{frame}

\maketitle

\setbeamerfont{subsection in toc}{size=\scriptsize}
\setbeamertemplate{section in toc}[sections numbered]

\begin{frame}{Table of Contents}
	\tableofcontents
\end{frame}


\section{Einführung in Auslastungsspiele}
% Zu langer Text?
\begin{frame}{Auslastungsspiel}
	\begin{block}{Szenario}
		\begin{itemize}
			\item $s-t$ Netzwerk $G$ mit Kostenfunktionen für jede Kante
			\item Fluss $f$ mit festen Wert soll von $s$ nach $t$ fließen
			\item Fluss wird von einem Kontinuum von Spielern kontrolliert
			\item Kantenkosten abhängig von der Auslastung dieser Kante
			\item Spieler versuchen Kosten zu optimieren
		\end{itemize}
	\end{block}

	\begin{block}{Später: schädliche Spieler}
		\begin{itemize}
			\item ein Teil des Flusses wird von schädlichen Spielern kontrolliert
			\item Ziel: Maximieren der Kosten
		\end{itemize}
	\end{block}
\end{frame}

% formatierung der Definitionen nicht einheitlich
\begin{frame}{Definition Auslastungsspiel}
	Ein symmetrisches, nicht-atomares \alert{Auslastungsspiel} $\mathcal G = (E, \vec{\ell}, \Pi, v)$
	\begin{itemize}
		\item $E$: Kantenmenge
		\item $\vec{\ell}$: Vektor von Latenzfunktionen $\ell_e : \mathbb R_+ \to \mathbb R_+$ für jede Kante $e$
		\item $\Pi$ Pfadmenge
		\item $v$ erzielter Flusswert
	\end{itemize}
\end{frame}

\begin{frame}{Fluss}
	Ein \alert{Fluss} ist eine Abbildung $f: \Pi \to \mathbb R^+$. 
	
	Der \alert{Wert} eines Flusses ist \[ \abs{f} = \sum_{P \in \Pi} f(P) \]
	
	Die Menge aller Flüsse mit Wert $v$ im Spiel $\mathcal G$ ist $F(\mathcal G, v)$.
	
	Die Auslastung an einer Kante $e$ ist \[ f(e) = \sum_{\mathclap{P \in \Pi: e \in P}} f(P) \]
\end{frame}

\begin{frame}{Kosten}
	Der \alert{Delay} auf einem Pfad $P \in \Pi$ ist
	\[ L(P) = \sum_{e \in P} \ell_e\left(f\left(e\right)\right). \]
	
	Die \alert{Kosten} eines Flusses $f$ sind
	\begin{align*}
		C(f) &= \sum_{P \in \Pi} f(P) \cdot L(P) \\
			&= \sum_{e \in E} f(e) \cdot \ell_e\left(f\left(e\right)\right). 
	\end{align*} 
\end{frame}

\begin{frame}{Beispiel (fortgesetzt)}
	\begin{tikzpicture}
		\SetGraphUnit{2}
		\GraphInit[vstyle=Classic]
		\SetUpEdge[style={->, thick}, color=gray]
		\tikzset{LabelStyle/.style = {fill=none}}
		\SetVertexNoLabel
		\Vertex{P1}
		\SO(P1){P2}
		
		\foreach \v in{1,2}{\EA[unit=3](P\v){Q\v}}
		
		\SetVertexLabel
		\WE[Lpos=180](P1){s}
		\EA(Q2){t}
		\foreach \v in{1,2}{\Edge[label=$x$, labelstyle={above}](P\v)(Q\v)}  %  P-Q Kanten mit l(x)
		
		
		\foreach \v in{1,2}{\Edge(s)(P\v)}  %  s-P Kanten
		\foreach \v in{1,2}{\Edge(Q\v)(t)}  %  Q-t Kanten
		% Rückkanten
		\Edge(Q1)(P2)
		
		% Fluss
		\draw [blue, ultra thick, ->, visible on=<2->] plot [smooth] coordinates{(s) (P1) (Q1) (P2) (Q2) (t)};
		\draw [green, ultra thick, ->, visible on=<3->] plot [smooth] coordinates{(s) (P2) (Q2) (t)};
	\end{tikzpicture}
\end{frame}

\begin{frame}{Nashfluss}
	Sei $f$ ein gültiger Fluss auf $\mathcal G$. 
	Der Fluss $f$ ist ein \alert{Nash-Fluss}, wenn für alle $P, P' \in \Pi$ und $f(P) > 0$ gilt
	\[
		L(P) \leq L(P').
	\]
	Für einen Nashfluss $f$:
	\begin{itemize}
		\item \emph{Nashkosten} $C(\mathcal G) := C(f)$
		\item \emph{Nashdelay} $D(\mathcal G) := D(f) := C(f)/|f|$
	\end{itemize}
	
	\begin{block}{Beobachtung}
		Jeder Nashfluss hat die gleichen Kosten $C(f)$. \\
		Für alle Pfade $P$ mit $f(P) > 0$ ist $L(P) = D(f)$, wenn $f$ Nashfluss.
	\end{block}
\end{frame}

\begin{frame}{Beispiel (fortgesetzt)}
	\begin{tikzpicture}
	\SetGraphUnit{2}
	\GraphInit[vstyle=Classic]
	\SetUpEdge[style={->, thick}, color=gray]
	\tikzset{LabelStyle/.style = {fill=none}}
	\SetVertexNoLabel
	\Vertex{P1}
	\SO(P1){P2}
	
	\foreach \v in{1,2}{\EA[unit=3](P\v){Q\v}}
	
	\SetVertexLabel
	\WE[Lpos=180](P1){s}
	\EA(Q2){t}
	\foreach \v in{1,2}{\Edge[label=$x$, labelstyle={above}](P\v)(Q\v)}  %  P-Q Kanten mit l(x)
	
	
	\foreach \v in{1,2}{\Edge(s)(P\v)}  %  s-P Kanten
	\foreach \v in{1,2}{\Edge(Q\v)(t)}  %  Q-t Kanten
	% Rückkanten
	\Edge(Q1)(P2)
	
	% Fluss
	\draw [blue, ultra thick, ->, visible on=<2->] plot [smooth] coordinates{(s) (P1) (Q1)  (t)};
	\draw [green, ultra thick, ->, visible on=<3->] plot [smooth] coordinates{(s) (P2) (Q2) (t)};
	\end{tikzpicture}
\end{frame}

\begin{frame}{Bekanntes über Auslastungsspiele}
	\begin{itemize}
		\item sind Potentialspiele
		\item pures Nashequilibrium existiert
	\end{itemize}
\end{frame}


\section{Bösartige Spieler}
\begin{frame}{Bösartige Spieler}
	\begin{itemize}
		\item kontrolliert Fluss $g$ mit Flusswert $w$
		\item maximiert Kosten der rationalen Spieler
		\item möglicherweise randomisiert mit W'keitverteilung $\gamma$
		\item wird als \emph{ein} Spieler modelliert
	\end{itemize}
\end{frame}

\begin{frame}{Induziertes Spiel}
	Sei $\mathcal G = (E, \vec{\ell}, \Pi, v)$, $\gamma$ eine W'keitverteilung auf $F(\mathcal G, w)$ und Fluss $g$ eine Probe aus $\gamma$. 
	
	Dann ist $\mathcal G^\gamma = (E, \vec{\ell^\gamma}, \Pi, v-w)$ ein \alert{induziertes Spiel} mit 
	\[ \ell_e^\gamma(x) = \mathbb E[\ell_e(x+ g(e))] .\]
\end{frame}

\begin{frame}{Beispiel (fortgesetzt)}
	Frame
\end{frame}

\begin{frame}{Malicious Best Response}
	Für einen Fluss $f$ (der rationalen Spieler) im Spiel $\mathcal G$ ist die \alert{bösartige beste Antwort} zu $f$ die Menge der Flüsse
	\[ MBR(f) = \argmax_{g \in F(\mathcal G, w)} \, C^g(f) .\]
	
	Eine W'keitsverteilung $\gamma$ auf $F(\mathcal G, w)$ ist ein MBR zu $f$, wenn ein Träger von $\gamma$ in $MBR(f)$ liegt.
\end{frame}

\begin{frame}{Equilibrium}
	Für ein Spiel $\mathcal G$ mit einem bösartigen Spieler ist $(f, \gamma)$ ein (semi-pures) \alert{Equilibrium}, wenn $f$ ein Nashfluss im induziertem Spiel $\mathcal G^\gamma$ und $\gamma$ eine MBR zu $f$ ist. 
	
	Das Equilibrium ist \alert{pur}, wenn $\gamma$ nur einen Träger besitzt.
	
	Nashdelay $D(\mathcal G, w)$ ist
	\[
		\sup \left\{\frac{C^\gamma(f)}{|f|} : (f,\gamma) \text{ ist ein Equilibrium} \right\}
	\]
\end{frame}

\begin{frame}{Beispiel (fortgesetzt)}
	Frame
\end{frame}

\begin{frame}{Price of Malice}
	Sei $\mathcal G = (E, \vec{\ell}, \Pi, v)$ ein Spiel mit einem bösartigen Spieler.
	Dann ist der \alert{Price of Malice} definiert durch
	\begin{align*}
		POM(f) &= \lim_{\epsilon \to 0^+} \dfrac{D(\mathcal G, \epsilon v) - D(\mathcal G)}{\epsilon D(\mathcal G)}  \\
		&= \frac{1}{D(\mathcal G)} \cdot \dv{\epsilon}(D(\mathcal G, v \epsilon))_{(\epsilon = 0)}.
	\end{align*}
\end{frame}

\begin{frame}{Weiteres zu Equilbria}
	Bei Auslastungsspielen mit einem bösartigen Spieler 
	\begin{itemize}
		\item existiert immer ein \emph{semi-pures} Nashgleichgewicht, wobei der bösartige Spieler randomisieren darf
		\item existieren pure Nashgleichgewichte (PNE) \emph{nicht} immer
		\item mit \emph{konkaven} Latenzfunktionen existiert ein PNE
	\end{itemize}
\end{frame}

\section{Untere Schranke für den PoM}

\begin{frame}{Ursachen für einen hohen PoM}
	\begin{itemize}
		\item leicht 'verstopfbare' Kanten: Latenfunktion steigt sehr schnell
		\item lange Pfade
	\end{itemize}
\end{frame}

\begin{frame}{Relative Slope}
	Sei $\ell : [0,1] \to \mathbb R^+$ differenzierbar.
	Dann ist der \alert{relative Slope} $d$
	\[ d = \sup_{x \in [0,1]} \dfrac{\ell'(x)}{\ell(x)} x .\]
	Dies entspricht dem \emph{relativen Fehler} $\kappa_{rel} = \dfrac{\text{rel. } \Delta y}{\text{rel. } \Delta x}$.
	
	\begin{table}[]
		\begin{tabular}{@{}llll@{}}
			\toprule
			Funktion       & const. & $x^a$ & $e^{ax}$ \\ \midrule
			relative Slope & $0$      & $a$     & $a$        \\ \bottomrule
		\end{tabular}
	\end{table}
	\begin{theorem}
		Es existiert ein Spiel $\mathcal G$ mit $\mathcal O(m)$ Kanten, sodass der Price of Malice $d \cdot (m-1)$ beträgt, wobei $d$ der relative Slope der Kanten ist.
	\end{theorem}
\end{frame}

\begin{frame}{Netzwerk mit hohem PoM}
	\begin{figure}
	\begin{tikzpicture}
		\SetGraphUnit{2}
		\GraphInit[vstyle=Classic]
		\SetUpEdge[style={->, thick}, color=gray]
		\tikzset{LabelStyle/.style = {fill=none}}
		\SetVertexNoLabel
		\Vertex{P1}
		\SO(P1){P2}
		\SO(P2){P3}
		
		\foreach \v in{1,2,3}{\EA[unit=3](P\v){Q\v}}
		
		\SetVertexLabel
		\WE[Lpos=180](P2){s}
		\EA(Q2){t}
		\foreach \v in{1,2,3}{\Edge[label=$\ell(x)$, labelstyle={above}](P\v)(Q\v)}  %  P-Q Kanten mit l(x)
	
		
		\foreach \v in{1,2,3}{\Edge(s)(P\v)}  %  s-P Kanten
		\foreach \v in{1,2,3}{\Edge(Q\v)(t)}  %  Q-t Kanten
		% Rückkanten
		\Edge(Q1)(P2)
		\Edge(Q2)(P3)
	\end{tikzpicture}
	\caption{Beispielnetzwerk mit $m=3$}
	\end{figure}
	
	Flusswert $v(\mathcal G) = mZ$ mit $Z = \argmax \ell'(x)/\ell(x) \cdot x$ \\
	Flusswert vom bösartigen Spieler $w(\mathcal G) = \epsilon v = \epsilon m Z$
\end{frame}

\begin{frame}{Bösartiger Fluss $g$}
	Um die Kosten zu maximieren, nutzt der bösartige Spieler alle Kanten, deren Kostenfunktion nicht konstant ist (Auslastung $\epsilon m Z$):
	\begin{figure}
		\begin{tikzpicture}
		\SetGraphUnit{2}
		\GraphInit[vstyle=Classic]
		\SetUpEdge[style={->, thick}, color=gray]
		\tikzset{LabelStyle/.style = {fill=none}}
		\SetVertexNoLabel
		\Vertex{P1}
		\SO(P1){P2}
		\SO(P2){P3}
		
		\foreach \v in{1,2,3}{\EA[unit=3](P\v){Q\v}}
		
		\SetVertexLabel
		\WE[Lpos=180](P2){s}
		\EA(Q2){t}
		\foreach \v in{1,2,3}{\Edge[labelstyle={above}](P\v)(Q\v)}  %  P-Q Kanten mit l(x)
		
		
		\foreach \v in{1,2,3}{\Edge(s)(P\v)}  %  s-P Kanten
		\foreach \v in{1,2,3}{\Edge(Q\v)(t)}  %  Q-t Kanten
		% Rückkanten
		\Edge(Q1)(P2)
		\Edge(Q2)(P3)
		
		% Bösartiger Fluss
		\draw [red, ultra thick, ->] plot [smooth] coordinates{(s) (P1) (Q1) (P2) (Q2) (P3) (Q3) (t)};
		\end{tikzpicture}
	\end{figure}
	Induzierte Latenzfunktion $\ell^g(x) = \ell(x + \epsilon m Z)$
\end{frame}

\begin{frame}{Nashfluss $f$}
	Rationale Spieler kontrollieren $(1- \epsilon)mZ$ Fluss
	\begin{figure}
		\begin{tikzpicture}
		\SetGraphUnit{2}
		\GraphInit[vstyle=Classic]
		\SetUpEdge[style={->, thick}, color=gray]
		\tikzset{LabelStyle/.style = {fill=none, color=gray}}
		\SetVertexNoLabel
		\Vertex{P1}
		\SO(P1){P2}
		\SO(P2){P3}
		
		\foreach \v in{1,2,3}{\EA[unit=3](P\v){Q\v}}
		
		\SetVertexLabel
		\WE[Lpos=180](P2){s}
		\EA(Q2){t}
		\foreach \v in{1,2,3}{\Edge[label=$\ell(x + \epsilon v)$, labelstyle={above}](P\v)(Q\v)}  %  P-Q Kanten mit l(x)
		\foreach \v in{1,2,3}{\Edge(s)(P\v)}  %  s-P Kanten
		\foreach \v in{1,2,3}{\Edge(Q\v)(t)}  %  Q-t Kanten
		% Rückkanten
		\Edge(Q1)(P2)
		\Edge(Q2)(P3)
		
		%Pfad
		\draw [blue, ultra thick] plot [smooth] coordinates{(s) (P1) (Q1) (t)};
		\draw [blue, ultra thick] plot [smooth] coordinates{(s) (P2) (Q2) (t)};
		\draw [blue, ultra thick, ->] plot [smooth] coordinates{(s) (P3) (Q3) (t)};
		\end{tikzpicture}
	\end{figure}
	Auf den blauen Pfaden ist der Fluss $(1- \epsilon)Z$, sonst $0$. Damit ist die Auslastung an den mittleren Kanten: $\textcolor{blue}{f(e)} + \textcolor{red}{g(e)} = (1- \epsilon)Z + \epsilon m Z = Z + \epsilon (m-1) Z$
\end{frame}

\begin{frame}{Berechnung PoM}
	Auslastung der mittleren Kanten: $Z + \epsilon Z (m-1)$
	
	Damit ist die Latenz der blauen Pfade $L(P) = \ell (Z + \epsilon (m-1) Z)$.
	Da $f$ der einzige Nashflus ist, ist der Nashdelay $D(\mathcal G, v) = L(P)$.
	Für $\epsilon = 0$ erhalten wir den Nashdelay $D(\mathcal G) = \ell(Z)$.
	\begin{align*}
		POM(\mathcal G) &= \lim_{\epsilon \to 0^+} \dfrac{D(\mathcal G, \epsilon v) - D(\mathcal G)}{\epsilon D(\mathcal G)} \\
			&= \lim_{\epsilon \to 0^+} \dfrac{\ell (Z + \epsilon (m-1) Z) - \ell(Z)}{\epsilon \ell(Z)} \\
			&= \dfrac{(m - 1) Z \ell'(Z)}{\ell(Z)} \\
			&= d \cdot (m - 1)
	\end{align*}
\end{frame}


\section{Windfall of Malice}
\begin{frame}{Beispielgraph mit negativer PoM}
	\begin{columns}[T]
		\begin{column}{0.6\textwidth}
			\begin{tikzpicture}
				\SetGraphUnit{2}
				\GraphInit[vstyle=Classic]
				\SetUpEdge[style={->, thick}, color=gray]
				\tikzset{LabelStyle/.style = {fill=none, sloped, above}}
				
				% Knoten
				\Vertices[unit=3]{circle}{t,p,s,q}
				\EA[Lpos=180](s){m}
				\EA[unit=2](m){n}
				
				% Kanten
				\Edge[label=$D_1(x)$](s)(p)
				\Edge[label=$D_1(x)$](q)(t)
				\Edge[label=$1$](s)(q)
				\Edge[label=$1$](p)(t)
				\Edge[label=$0$](p)(n)
				\Edge[label=$0$](m)(n)
				\Edge[label=$0$](m)(q)
				\Edge[label=$D_2(x)$](p)(m)
				\Edge[label=$D_2(x)$](n)(q)
			\end{tikzpicture}
		\end{column}
	
		\begin{column}{0.4\textwidth}
			$D_1(x) = x^d/2$ \\
			$D_2(x) = (2x)^d/2$
			
			Flusswert $v = 1$ \\
			Bösartiger $w = \epsilon$
			
		\end{column}
	\end{columns}
\end{frame}

\begin{frame}{Bösartiger Fluss}
	\begin{tikzpicture}
		\SetGraphUnit{2}
		\GraphInit[vstyle=Classic]
		\SetUpEdge[style={->, thick}, color=gray]
		\tikzset{LabelStyle/.style = {fill=none, sloped, above, color=gray}}
		
		% Knoten
		\Vertices[unit=3]{circle}{t,p,s,q}
		\EA[Lpos=180](s){m}
		\EA[unit=2](m){n}
		
		% Kanten
		\Edge[label=$D_1(x)$](s)(p)
		\Edge[label=$D_1(x)$](q)(t)
		\Edge[label=$1$](s)(q)
		\Edge[label=$1$](p)(t)
		\Edge[label=$0$](p)(n)
		\Edge[label=$0$](m)(n)
		\Edge[label=$0$](m)(q)
		\Edge[label=$D_2(x)$](p)(m)
		\Edge[label=$D_2(x)$](n)(q)
		
		% Bösartiger Fluss
		\draw [red, ultra thick, ->] plot [smooth] coordinates{(s) (p) (m) (n) (q) (t)};
	\end{tikzpicture}
\end{frame}

\begin{frame}{Nashflow der rationalen Spieler}
	\begin{columns}[T]
		\begin{column}{0.5\textwidth}
			\begin{figure}
				\begin{tikzpicture}
				\SetGraphUnit{2}
				\GraphInit[vstyle=Classic]
				\SetUpEdge[style={->, thick}, color=gray]
				\tikzset{LabelStyle/.style = {fill=none, sloped, above, color=gray, font=\sffamily\footnotesize}}
				
				% Knoten
				\Vertices[unit=3]{circle}{t,p,s,q}
				\EA[Lpos=180](s){m}
				\EA[unit=2](m){n}
				
				% Kanten
				\Edge[label=$D_1(x)$](s)(p)
				\Edge[label=$D_1(x)$](q)(t)
				\Edge[label=$1$](s)(q)
				\Edge[label=$1$](p)(t)
				\Edge[label=$0$](p)(n)
				\Edge[label=$0$](m)(n)
				\Edge[label=$0$](m)(q)
				\Edge[label=$D_2(x)$](p)(m)
				\Edge[label=$D_2(x)$](n)(q)
				
				% Auslastungen
				\tikzset{LabelStyle/.style = {fill=none, sloped, below, color=blue}}
				\Edge[label=$a(\epsilon)$](s)(p)
				\Edge[label=$a(\epsilon)$](q)(t)
				\Edge[label=$a(\epsilon) - 2 b(\epsilon)$](s)(q)
				\Edge[label=$a(\epsilon) - 2 b(\epsilon)$](p)(t)
				\Edge[label=$b(\epsilon)$](p)(n)
				
				\Edge[label=$b(\epsilon)$](m)(q)
				\Edge[label=$b(\epsilon)$](p)(m)
				\Edge[label=$b(\epsilon)$](n)(q)
			\end{tikzpicture}
			\end{figure}
		\end{column}
		\begin{column}{0.5\textwidth}
			$1-\epsilon$ Fluss zu verteilen
			
			 Für $\epsilon = 0$:
			 $a(\epsilon) = 1, b(\epsilon) = 1/2$
			 $D(\mathcal G) = 1.5$
			
		\end{column}
	\end{columns}
	
\end{frame}


\section{Abschluss}
\begin{frame}{Zusammenfassung}
	\begin{itemize}
		\item 
	\end{itemize}
\end{frame}

\begin{frame}{Offene Fragen}
	\begin{itemize}
		\item Strikte Schranken für 
		\begin{itemize}
			\item Price of Malice
			\item Windfall of Malice
		\end{itemize}
		\item Charakterisierung von Netzwerken
		\item Ist das Finden von Equilibria PPAD-complete?
	\end{itemize}
\end{frame}
\end{document}